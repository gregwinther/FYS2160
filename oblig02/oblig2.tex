\documentclass[10pt,a4paper]{amsart}

\usepackage{amsmath}
\usepackage{physics}
\usepackage{listings}
\usepackage{graphicx}
\usepackage[]{hyperref}
\usepackage{float}

\lstset{
	frame = single,
	language = Python,
	showstringspaces = false,
	tabsize = 2
}

\title[Rotation of Diatomic Molecules]{Rotation of Diatomic Molecules \\
	\hrulefill\fbox{FYS2160}\hrulefill}
	
\author[Winther-Larsen]{Sebastian G. Winther-Larsen\\
\href{https://github.com/gregwinther/FYS2160/}{\texttt{github.com/gregwinther}}}

\date{\today}

\begin{document}

\maketitle

\section{Introduction}
This study is an exercise in how to compute all interesting properties of a simple system consisting of diatomic molecules A general method can be applied to all such systems. First, finding the partition function. Second, one computes the average energy of the system. Lastly, the remaining interesting aspects of the system can be found, like the entropy and heat capacity. Herein, a simple model system is studied first, and then similar methods are applied to a more general system. Both analytical and numerical methods are used on the latter. All program code for this project can be found at the github address stated above.

\section{A simplified model system}
In this simple system we look at a diatomic molecule, which at low temperatures can be in four different states, $i=1,2,3,4$, with energies $\varepsilon_i=\varepsilon$, $\varepsilon_2=\varepsilon_3=\varepsilon_4=2\varepsilon$. In other words, this system has two possible energies, $\varepsilon$ and $2\varepsilon$. The highest energy has a degeneracy of $3$.

The partition function is found by the following formula
\begin{equation}
Z = \sum_ie^{-\beta E_i},\quad \beta=\frac{1}{kT},
\end{equation}
colloquially, the sum of a special transform called the Boltzmann factor of every state. For this system the partition function is
\begin{equation}
Z = e^{-\beta\varepsilon}+3e^{-2\beta\varepsilon}
\end{equation}
One could employ Hermholtz' free energy
\begin{equation}
F = E - TS= -kT\ln Z.
\end{equation}
First, computing
\begin{align*}
\ln Z &= \ln (e^{-\beta\varepsilon}+3e^{-2\beta\varepsilon} ) = \ln( e^{-\beta\varepsilon}) + \ln(1+3e^{-\beta\varepsilon} ) \\
&\approx ln( e^{-\beta\varepsilon}) + \ln(3e^{-\beta\varepsilon} )  = -\beta\varepsilon + \ln(3) -\beta\varepsilon  = -2\beta\varepsilon + \ln(3).
\end{align*}
The approximation on the second line might not strictily speaking be necessary because the expression for the partition function is simple enough as it is. Instead of going this way I will compute the average energy instead.

The average energy of this system can be found by differentiating $Z$ with respect to $\beta$, and multiplying by $-1/Z$.
\begin{equation}
\bar{E} = -\frac{1}{Z}\frac{\partial Z}{\partial \beta}=
\frac{\varepsilon e^{-\varepsilon\beta}+6\varepsilon e^{-2\varepsilon\beta} }{e^{-\varepsilon\beta} + 3e^{-\varepsilon\beta}} = \frac{\varepsilon(e^{exp(\varepsilon\beta}+6)}{e^{\varepsilon\beta}+3}.
\end{equation}
Because $\beta = 1/kT$ we have the energy as a function of temperature.

The heat capaciy, when no work is done on the system and volume is constant, is given by
\begin{equation}
C_V = \left(\frac{\partial E}{\partial T} \right)_V
\end{equation}
In this case, the heat capacity is 
\begin{equation*}
C_V =\left(\frac{\partial E}{\partial T} \right)_V =
\frac{3\varepsilon e^{\varepsilon/Tk}}{T^2k(e^{\varepsilon/Tk}+3)^2}
\end{equation*}

A plot of the heat capacity can be found in figure \ref{fig:heatcap1}. We see that the heat capacity quickly approaches zero for larger temperatures. This result is further underlined by the results shown in table \ref{tab:heatcap1}.

\begin{table}
	\caption{Heat capacity for the simple diatomic molecule system.}
	\begin{tabular}{rr} \hline
	T[K]  & C[J/K] \\ \hline 
	$  1$ & $8.8114\times10^5$ \\
	$ 25$ & $2.1152\times10^7$ \\
	$ 50$ & $5.1519\times10^8$ \\
	$100$ & $1.2707\times10^8$ \\
	$273$ & $1.6902\times10^9$ \\ \hline
	\end{tabular}
	\label{tab:heatcap1}
\end{table}

This system is not a very realistic system. There are very few moleucules involved, and the volume of the sytem is assumed to be unchanged as the temperature rises. One would assume that a system of gasseous nature would expand as the temperature increases. Furthermore, the temperature domain of the system is just around a few degrees Kelvin at which most, if not all, substances are liquid or solid. There are very few systems able to sustain temperatures this low\footnote{One example may be liquid helimum, which has a temperature of around 4 Kelvin}. This is a good example of the third law of thermodynamics.

\begin{figure}
	\includegraphics[width=0.9\textwidth]{figures/heatcapsimple.png}
	\caption{Plot of heat capacity vs temperature for the simple 			diatomic molecule system.}
	\label{fig:heatcap1}
\end{figure}

\section{Full model system of a rotating dimolecule}

For a diatomic molecule, the rotational energies are quantized inte energy levels described by $j$
\begin{equation}
\varepsilon_j=j(j+1)\theta_r k, \quad j=0,1,2,\dots
\end{equation}
where
\begin{equation}
\theta_r k= \frac{\hbar^2}{2I}
\end{equation}
where $I$ is the moment of intertia for the molecule. The values of $\theta_r$ for some molecules are listed in table \ref{tab:thetar}. The energy states are degenerate, and the degeneration of each energy given by $j$ is $d(j)=2j+1$.

\begin{table}
	\centering
	\caption{$\theta_r$ for different molecules}
	\begin{tabular}{ccccccc} \hline
 	 & $H_2$ & $HCl$ & $HI$ & $N_2$ & $Cl_2$ & $I_2$ \\ \hline
	$\theta_r(K)$ & 85.4 & 15.2 & 9.0 & 2.86 & 0.346 & 0.054 \\ \hline
	\end{tabular}
	\label{tab:thetar}
\end{table} 

\subsection{Analytical approach}

The partition function for this system is
\begin{equation}
\label{eq:partition}
Z_{rot} = \sum_j (2j + 1)e^{-\beta\varepsilon_j} 
        = \sum_j (2j + 1)e^{-\beta j(j+1)\theta_r k}
        = \sum_j (2j + 1)e^{-j(j+1)\theta_r/T}
\end{equation}
Figure \ref{fig:partitionterms} shows the terms in the sum of this function for various values of $T/\theta_r$. The tendency ic clear: as the temperature $T$ gets higher relative to $theta_r$, the term $z(j)$ is more significant, especially as $j$ increases as well. If $T$ is small, relative to $\theta_r$, $z(t)$ quickly drops to zero for higher states $j$. 

\begin{figure}
	\centering
	\includegraphics[width=0.9\textwidth]{figures/partitionterms.png}
	\caption{The terms in the partition function in equation \ref{eq:partition} for various values of $\frac{T}{\theta_r}$}
	\label{fig:partitionterms}
\end{figure}

The problem with the partition function in equation \ref{eq:partition} is that it difficult to evaluate a sum like this in closed form exactly. But for a sufficiently large $T/\theta_r$ one can approximate the sum with an integral.
\begin{equation}
Z_{rot} \approx \int_0^{\infty} (2j+1)e^{-j(j+1)\theta_r/T}dj
\end{equation}
This integral is realtively easy to solve by the substitution $x=j(j+1)\theta_r/T$ and $dx = (2j+1)\theta_r dj/T$ which gives
\begin{equation}
\frac{T}{\theta_r}\int_0^{\infty}e^{-x}dx=\frac{T}{\theta_r}
\end{equation}
Well, what do you know? A (perhaps not so) suprisingly simple and nice expression fell out.

On the other hand, we have the opposite situation where $T << \theta_r$. It is sufficient to only look at the first few terms within the sum of the partition function (equation \ref{eq:partition}) in this case (e.g. $j=0,1,2,3$)
\begin{equation}
Z_{rot} \approx 1+e^{-\theta_r/T}+3e^{-2\theta_r/T}+5e^{-6\theta_r/T}+7e^{-12\theta_r/T}
\end{equation}
which for the borderline case $T=\theta_r$ will form a convergent series\footnote{This is easy to check with the ratio test, where one would formulate the test limit $L=\lim_{n\to\infty}\abs{a_{n+1}/a_n}$. If $L<1$ the series converges absolutely.}
\begin{equation*}
Z_{rot} \approx 1 + \frac{3}{e^2}+\frac{5}{e^6}+\frac{7}{e^{12}} + \dots \approx 1 + 0.406006 + 0.012394 + 0.0000430 + \dots \approx 1.41844
\end{equation*}
while for the case $T >> \theta_r$, all terms except the first one will disappear and we get
\begin{equation*}
Z_{rot} = 1
\end{equation*}

The average energy of the system can be calculated in the same way as for the simplified system in the previous section, with the following equation
\begin{equation}
\bar{E}=-\frac{1}{Z}\frac{\partial Z}{\partial\beta}, \quad \beta=\frac{1}{kT}
\end{equation}
In our situation we must have that $T = 1/k\beta$. Then we get the following for the high temperature system, where $Z = T/\theta_r$
\begin{equation*}
\label{eq:anale}
\bar{E}=-\frac{1}{Z}\frac{\partial Z}{\partial \beta } = 
-\frac{\theta_r}{T}\frac{\partial}{\partial\beta}\frac{T}{\theta_r} = 
-\frac{1}{T}\frac{\partial}{\partial\beta}\frac{1}{k\beta}=\frac{1}{T}\frac{k}{\beta^2}=\frac{k^2T^2}{kT}=kT
\end{equation*}
While for the low temperature system, the average energy must be zero, because the partition function is a constant.

The heat capacity, assuming that the volume of the system is constant, is given by
\begin{equation}
C_V(T) = \left(\frac{\partial E}{\partial T}\right)_V
\end{equation}
which for the high temperature system gives us a heat capacity of Boltzmann's constant $k$. At a low temperature, the heat capacity is zero, in accordance with the simple system in the previous section. Furthermore, this is in accordance with \emph{the third law of thermodynamics}. At zero temperature, a system should settle into its unique lowest-energy state. where
\begin{equation}
\label{eq:thirdlaw}
C_v \to 0 \text{ as } T \to 0
\end{equation}

\subsection{Numerical approach}
In order to get a fuller understanding of the diatomic system, similar calculations like the ones above will be done, but now numerically.

The following python function evaluates the partition function for various $T/\theta_r$.
\lstinputlisting[firstline=11, lastline=35]{scripts/partition.py}
The function will continue to compute new terms of the sum \lstinline|Z_| until these terms are lower than a tolerance $\epsilon=1.0\times10^{-12}$. A plot of the value of the partition function for values of $T/\theta_r$ from $1.0\times10^{-3}$ to $1.0\times10^{3}$ is shown in figure \ref{fig:numpartition}. It is evident, that for temperatures $T << \theta_r$ not much is happening in the system as the partition function asymptotically approaches a low value.

\begin{figure}
	\centering
	\includegraphics[width=0.9\textwidth]{figures/numpartition.png}
	\caption{Numerical computiation of the partition function plotted against various values of $T/\theta_r$. The figure has a logarithmic scale on the x-axis.}
	\label{fig:numpartition}
\end{figure}

It would be interesting to see how the numerical computation compares with the analytacl approximation. Such a comparison i shown in figure \ref{fig:numvsanal}. We can see that there is some difference between the two functions, especially at low levels. This makes sense because one of the assumptions that was made in order to find an analytical approximation was that $T >> \theta_r$. we see that the two solutions converge as the temperature $T$ increases relative to $\theta_r$. The values plotted in figure \ref{fig:numvsanal} are absolute values, the relative error will be insignificat at high temperatures. For high $T$, the analytical approximation misses by $1/3$.

\begin{figure}
	\centering
	\includegraphics[width=0.9\textwidth]{figures/numvsanal.png}
	\caption{Absolute difference between the numerical computation and the analytical computation of the partition function. The analytical solution improves for higher temperatures, according to assumption.}
	\label{fig:numvsanal}
\end{figure}

The energy and the heat capacity can also be computed numerically. In this system we have an average energy of
\begin{equation}
\label{eq:numenergy}
\bar{E} = -\frac{1}{Z}\frac{\partial Z}{\partial \beta} \approx -\frac{1}{Z}\frac{\Delta Z}{\Delta \beta}
\end{equation} 
elaborating on this
\begin{equation}
\label{eq:deltabeta}
\Delta \beta = \Delta \frac{1}{kT}= \frac{1}{k}\left(\frac{1}{T_{i+h}}-\frac{1}{T_i} \right)
\end{equation}
where $h$ is a tiny step. Putting equation \ref{eq:deltabeta} into equation \ref{eq:deltabeta} we get something like
\begin{equation}
\frac{\bar{E}_i}{k}=-\frac{k}{Z_i}\frac{Z_{i+h}-Z_i}{\frac{1}{T_{i+h}}-\frac{1}{T_i}}
\end{equation}

A functiong that attempts this computation would look something like this
\lstinputlisting[firstline=37, lastline=55]{scripts/partition.py}

A plot of the results from the program listing above is shown in figure \ref{fig:numenergy}. The plot forms a straight line, which is in accordance with the analytical result in equation \ref{eq:anale}.

\begin{figure}
	\centering
	\includegraphics[width=0.9\textwidth]{figures/numenergy.png}
	\caption{Numerically computed average energy of the system.}
	\label{fig:numenergy}
\end{figure}

Employing the same finite difference scheme to compute the heat capacity would only result in a straight line, but would be done in a similar manner by computing
\begin{equation}
C_{V,i}=\frac{E_{i+h}-E_i}{T_{i+1}-T_i}
\end{equation}

\end{document}