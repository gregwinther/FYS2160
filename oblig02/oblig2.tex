\documentclass[10pt,a4paper]{amsart}

\usepackage{amsmath}
\usepackage{physics}
\usepackage{listings}
\usepackage{graphicx}
\usepackage[]{hyperref}

\lstset{
	frame = single,
	language = Python,
	showstringspaces = false,
	tabsize = 2,
	otherkeywords = {self},
	keywordstyle = \color{blue},
	identifierstyle=\color{deepgreen},
 	stringstyle=\color{orange},
 	backgroundcolor=\color{mygray}
}

\title[Rotation of Diatomic Molecules]{Rotation of Diatomic Molecules \\
	\hrulefill\fbox{FYS2160}\hrulefill}
	
\author[Winther-Larsen]{Sebastian G. Winther-Larsen\\
\href{https://github.com/gregwinther/FYS2160/}{\texttt{github.com/gregwinther}}}

\date{\today}

\begin{document}

\maketitle

\tableofcontents

\section{Introduction}
This study shows how to connect a microscopic and macroscopic representation of a canonical system, as system with given $N$, $V$ and $T$ (number of particles/molecules, volume of the system and temperature respectively). A general method can be applied to all canonical systems. First, finding the partition function. Second, one derives a function for Hermholtz free energy. Lastly, the remaining interesting aspects of the system can be found, like the entropy and heat capacity.

\section{A simplified model system}
In this simple system we look at a diatomic molecule, which at low temperatures can be in four different states, $i=1,2,3,4$, with energies $\varepsilon_i=\varepsilon$, $\varepsilon_2=\varepsilon_3=\varepsilon_4=2\varepsilon$. In other words, this system has two possible energies, $\varepsilon$ and $2\varepsilon$. The highest energy has a degeneracy of $3$.

The partition function is found by the following formula
\begin{equation}
Z = \sum_ie^{-\beta E_i},\quad \beta=\frac{1}{kT},
\end{equation}
colloquially, the sum of a special transform called the Boltzmann factor of every state. For this system the partition function is
\begin{equation}
Z = e^{-\beta\varepsilon}+3e^{-2\beta\varepsilon}
\end{equation}
One could employ Hermholtz' free energy
\begin{equation}
F = E - TS= -kT\ln Z.
\end{equation}
First, computing
\begin{align*}
\ln Z &= \ln (e^{-\beta\varepsilon}+3e^{-2\beta\varepsilon} ) = \ln( e^{-\beta\varepsilon}) + \ln(1+3e^{-\beta\varepsilon} ) \\
&\approx ln( e^{-\beta\varepsilon}) + \ln(3e^{-\beta\varepsilon} )  = -\beta\varepsilon + \ln(3) -\beta\varepsilon  = -2\beta\varepsilon + \ln(3).
\end{align*}
The approximation on the second line might not strictily speaking be necessary because the expression for the partition function is simple enough as it is. Instead of going this way I will compute the average energy instead.

The average energy of this system can be found by differentiating $Z$ with respect to $\beta$, and multiplying by $-1/Z$.
\begin{equation}
\bar{E} = -\frac{1}{Z}\frac{\partial Z}{\partial \beta}=
\frac{\varepsilon e^{-\varepsilon\beta}+6\varepsilon e^{-2\varepsilon\beta} }{e^{-\varepsilon\beta} + 3e^{-\varepsilon\beta}} = \frac{\varepsilon(e^{exp(\varepsilon\beta}+6)}{e^{\varepsilon\beta}+3}.
\end{equation}
Because $\beta = 1/kT$ we have the energy as a function of temperature.

The heat capaciy, when no work is done on the system and volume is constant, is given by
\begin{equation}
C_V = \left(\frac{\partial E}{\partial T} \right)_V
\end{equation}
In this case, the heat capacity is 
\begin{equation*}
C_V =\left(\frac{\partial E}{\partial T} \right)_V =
\frac{3\varepsilon e^{\varepsilon/kT}}{T^2k(e^{\varepsilon/Tk}+3)^2}
\end{equation*}

\end{document}