\documentclass[11pt,twocolumn]{amsart}

\usepackage{amsmath}
\usepackage{physics}
\usepackage[margin=0.5in]{geometry}

\setcounter{section}{-1} 

\begin{document}
\section{Physical constants}
$k=1.381\times10^{-23}J/K = 8.617\times10^{-5}eV/K$, $N_A = 6.022\times10^{23}$, $R=8.315 J/mol\cdot K$, $h=6.626\times10^{-34}J\cdot s = 4.136\times10^{-15}eV\cdot S$

\section{Energy in Thermal Physics}

\subsection{Thermal equilibrium}
\emph{Temperature} is a measure of the tendency of an object to spontaneously give up energy to its surroundings. When two objects are in thermal contact, the one that tends to spontaneously \emph{lose} energy is at the \emph{higher} energy. Room temperature $~300K$

\subsection{The ideal gas} $PV = nRT = Nk_BT$. $n$ is no of moles, $N=nN_A$ is number of molecules. $k_B = R/N_A$. Latter equation is valid when avg. space b/w molecules is larger than size of molecules. $\bar{E}_{K,trans} = \frac{3}{2}kT$.

\subsection{Equipartition of energy}
Theorem: at temperature $T$, the average energy of any quadratic degree of freedom is $\frac{1}{2}kT$. $U_{thermal} = Nf\frac{1}{2}kT$. Monoatomic gas: $f=3$. Diatomic gas: $f=5,6$ (3 trans., 2-3, rot.) or $f=8$ (3 trans., 3 rot., 2 vibr. K, P). Solid: $f=6$ (6 vibr. 3K, 3P). Some vibrational energies may be "frozen out" at room temperature.

\subsection{Heat and work}  
First law of thermodynamics \\$\Delta U = Q + W$. The change in energy is equal to the heat added and the work done. Heat transfer happens by \emph{conduction}, \emph{convection} and \emph{radiation}.

\subsection{Compression work}
Consider a piston. The force is $F=PA$. Assumes that the pressure is uniform. Compression must be slow enough so the gas has time to continually equilibrate to the changing conditions $\rightarrow$ \emph{quasistatic}. A compressed gas, i.e. negative $\Delta V$ gives $W=F\Delta x=PA\Delta x = -P\Delta V$. 
\subsubsection{Compression of ideal gas}
Two idealised ways: \\ \emph{Isothermal} compression is so slow that the temperature of the gas does not rise (quasistatic). \emph{Adiabatic} compression is so fast that no heat escapes during the compression. $VT^{f/2}=\text{constant}$, $V^{\gamma}P=\text{constant}$. $\gamma=\frac{f+2}{f}$ is the adiabatic exponent.

\subsection{Heat capacities}
Amount of heat needed to raise an object's temperature, per degree temperature increase: $C = \frac{Q}{\Delta T} = \frac{\Delta U - W}{\Delta T} $. $W=0$ and $V =\text{constant}$ is called heat capacity heat capacity at \emph{constant volume}, else there would be compression work, $-P\Delta V$. $C_V=\left(\frac{\partial U}{\partial T}\right)_V$. If an object expand when heated and do work on surroundings, there is negative $W$. At constant $P$, $Q$ i unambiguous $\rightarrow$ heat capacity at \emph{constant pressure}: $C_P = \left(\frac{\Delta U -(-P\Delta V)}{\Delta T}\right)_P=\left(\frac{\partial U}{\partial T}\right)_P + P\frac{\partial V}{\partial T}_P$.
\subsubsection{Latent heat} During a face transformation $C = \frac{Q}{\Delta T} = \frac{Q}{0} = \infty$. While $L=\frac{Q}{m}$ is the heat required to accomplish the transformation, the \emph{latent heat}.
\subsubsection{Enthalpy} To create a rabbit out of nothing, the sorcerer must summon up not only the energy $U$ of the rabbit, but also some additional energy, equal to $PV$, to push the atmosphere out of the way to make room. The \emph{enthalpy}: $H= U + PV$. At constant $P$, $\Delta H = Q + W_{other}$.

\section{The Second Law}
Entropy, and multiplicity, tends to increase.
\subsection{Two-state systems} 
Multiplicity is given by the binomial coefficient. $\Omega(N,n) = \frac{N!}{n!\cdot (N-n)! } = \binom{N}{n}$. How many ways to pick $n$ objects out of $N$. Permutations: $_nP_k = n(n-1)(n-2)\dots (n-k) = \frac{n!}{(n-k)!} $. Unordered permutations: $_nC_k = _nP_k / k! = \binom{n}{k}$.
\subsection{Einstein model of a solid} 
One energy unit is $h\nu = \hbar\omega$. Multiplicity of Einstein solid with $N$ oscillators (N/3 atoms) and $q$ energy units: $\Omega(N,q) = \binom{q+N-1}{q}$.
\subsection{Interacting systems}
Two solids are \emph{weakly coupled} when flow of energy between them is much slower than flow of energy between atoms within each solid. Macrostate is the combined system, specified by temporarily constrained values $U_A$, $U_B$. Over time they will change, with the sum $U_{tot} = U_A + U_B$ remaining fixed. All parameters in such a system is $N_A$, $N_B$, $q_{tot}=q_A+q_B$, $\Omega_{tot}=\Omega_A \Omega_B$. Fundamental assumption of statistical mechanics: In an isolated system, all accessible microstates are equally probable.
\subsection{Large systems}
If $\abs{x} << 1$, a Taylor expansion gives $\ln(x+1)\approx x$. If $N >> 1$ one can apply \emph{Stirling's approximation}: $N! \approx N^Ne^{-1}\sqrt{2\pi N}$. If $N$ is a large number, and $N!$ is very large, the square root factor can be omitted. This is usually good enough: $\ln N! = N\ln N - N$. In a large Einstein solid $q>>N$ is the high temperature limit: $\Omega \approx \frac{(q+N)!}{q!N!}$, $\ln\Omega \approx (q+N)\ln(q+N) - q\ln q - N\ln N$, where $\ln(q+N) \approx \ln q + \frac{N}{q}$. S.T.: $\ln\Omega \approx N\ln \frac{q}{N} + N + \frac{N^2}{q}$.
\subsection{The ideal gas}
$\Omega(U,V,N) = f(N)V^NU^{3N/2}$, where $f(N)$ is a complicated function of $N$. $\Omega_N = \frac{1}{N!}\frac{V^N}{h^{3N}}\times\text{(area of momentum hypersphere)}$. $\text{``area''} = \frac{2\pi^{d/2}}{(\frac{d}{2}-1)}r^{d-1}$, where  in general $d=3N$ and $r = \sqrt{2mU}$.
\subsection{Entropy}
$S = k\ln\Omega$. Now you see why the logarithm of the multiplicity is nice to have. Entropy of an ideal gas: $S = Nk \left[\ln\left(\frac{V}{N}\left(\frac{4\pi mU}{3Nh^2}\right)^{3/2} \right) + \frac{5}{2} \right]$ (The Sackur-Tetrode equation). Depends on $V$, $E$, $N$. Increasing any of them increases $S$. 
\subsubsection{Mixing} One gas into another chamber: $\Delta S_A = Nk\ln\frac{V_f}{V_i}=Nk\ln 2$. Two gases mixing, by removing a partition: $\Delta S_{tot} = \Delta S_A + \Delta S_B = 2Nk \ln 2$. Must be distinguishable gases. Gibbs Paradox.
\subsubsection{Irreversible} Processes that create new entropy are said to be irreversible. A sudden expansion is irreversible. A reversible volume change must in fact be quasistatic S.T. $W = - P\Delta V $.

\section{Interactions and Implications}
\subsection{Temperature} Two Einstein solids: $\frac{\partial S_A}{\partial U_A} = \frac{\partial S_B}{\partial U_B}(=0)$ at equilibrium with $N_A$, $N_B$ fixed.


\end{document} 